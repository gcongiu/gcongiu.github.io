\documentclass[a4paper,10pt]{article}

%A Few Useful Packages
\usepackage{marvosym}
\usepackage{fontspec} %for loading fonts
\usepackage{xunicode,xltxtra,url,parskip} %other packages for formatting
\RequirePackage{color,graphicx}
\usepackage[big]{layaureo} %better formatting of the A4 page
\usepackage{supertabular} %for Grades
\usepackage{titlesec} %custom \section

%Setup hyperref package, and colours for links
\usepackage{hyperref}
\usepackage{array, booktabs}
\usepackage{longtable, tabu}
\usepackage[table]{xcolor}
\usepackage{ragged2e}

\definecolor{linkcolour}{rgb}{0,0.2,0.6}
\hypersetup{colorlinks,breaklinks,urlcolor=linkcolour, linkcolor=linkcolour}
\usepackage{longtable}

%FONTS
\defaultfontfeatures{Mapping=tex-text}
\setmainfont[
SmallCapsFont = Fontin-SmallCaps.otf,
BoldFont = Fontin-Bold.otf,
ItalicFont = Fontin-Italic.otf
]
{Fontin.otf}

\titleformat{\section}{\Large\scshape\raggedright}{}{0em}{}[\titlerule]
\titlespacing{\section}{0pt}{3pt}{3pt}

%Italian hyphenation for the word: ''corporations''
\hyphenation{im-pre-se}

%-------------WATERMARK TEST [**not part of a CV**]---------------
\usepackage[absolute]{textpos}

\setlength{\TPHorizModule}{30mm}
\setlength{\TPVertModule}{\TPHorizModule}
\textblockorigin{2mm}{0.65\paperheight}
\setlength{\parindent}{0pt}

%--------------------BEGIN DOCUMENT----------------------
\begin{document}

\pagestyle{empty} % non-numbered pages

\font\fb=''[cmr10]'' %for use with \LaTeX command

%--------------------TITLE-------------
\par{\centering
    {\Huge \textsc{Giuseppe Congiu}
}\bigskip\par}

%--------------------SECTIONS-----------------------------------
%Section: Personal Data
\section{Contact Information}

\begin{tabular}{l}
    Phone: +39 (350) 166-2487\\
    Email: giuseppe.congiu81@gmail.com\\
%    Webpage: \url{https://gcongiu.github.io}\\
%    Linkedin: \url{https://www.linkedin.com/in/giuseppecongiu81}
\end{tabular}

%Section: Education
\section{Education}
\begin{tabular}{rp{11cm}}
\textsc{7/12/2018} & Ph.D. in \textsc{Computer Science}\\
          & \textbf{Johannes Gutenberg University of Mainz}\\
          & Mark: Magna cum Laude\\
          & Dissertation: "Improving I/O Performance in HPC Through Guided\\
          & Prefetching and Non-Volatile Memory Devices"\\
          & \small Advisors: Prof. Dr. Andr\'e Brinkmann (JGU), Dr. Sai Narasimhamurthy (Seagate)\\

\textsc{12/17/2008} & M.S. in \textsc{Electrical \& Electronic Engineering},\\
          & \textbf{University of Cagliari}\\
          & Mark: 110/110\\
          & Thesis: ``Cell BE: Performance Analysis of the Element\\
          & Interconnect Bus and Development of an Alternative Packed\\
          & Switched Solution''\\
          & \small Advisor: Prof. Luigi \textsc{Raffo}\\

\textsc{10/24/2005} & B.S. in \textsc{Electrical \& Electronic Engineering} \\
          & \textbf{University of Cagliari}\\
          & Mark: 99/110\\
\end{tabular}

\section{Appointments}
%\begin{tabular}{rp{11cm}}
\begin{longtable}{rp{11cm}}
\textsc{01/08/2024 - present} & Senior Software Engineer at
    \textsc{NVIDIA, 2788 San Tomas Expressway, Santa Clara, CA 95051}\\
         & \textbf{\emph{Job Description:}}\\
         & \emph{\begin{itemize}
            \item Design, implement and maintain highly-optimized communication runtimes for Deep Learning frameworks (e.g. NCCL for TensorFlow/Pytorch) and HPC programming interfaces (e.g. UCX for MPI/OpenSHMEM) on GPU clusters.
            \item Participating in and contributing to parallel programming interface specifications like MPI/OpenSHMEM.
            \item Design, implement and maintain system software that enables interactions among GPUs and interactions between GPUs and other system components.
            \item Creating proof-of-concepts to evaluate and motivate extensions in programming models, new designs in runtimes and new features in hardware.
         \end{itemize}}\\
         & \emph{\textbf{Projects:} NVIDIA Collective Communication Library, NCCL}\\

\textsc{07/19/2021 - 01/05/2024} & Research Scientist at \textsc{Innovative Computing Laboratory, University of Tennessee, Suite 203 Claxton, 1122 Volunteer Blvd, Knoxville, TN 37996, USA}\\
         & \textbf{\emph{Job Description:}}\\
         & \emph{\begin{itemize}
             \item Work on the design and development of high-quality software (C/C++) for performance monitoring of new and advanced hardware and software technologies.
             \item Collaborate with industry partners, including AMD, HPE, IBM, Intel, NVIDIA and research institutions.
             \item Develop and coordinate test plans, use test suites, and perform tests on software solutions.
             \item Utilize Continuous Testing with Jenkins and other systems.
             \item Implement and maintain build processes using Spack, CMake, and Makefile.
             \item Integrate software with other systems, test interfaces, and manage version control.
             \item Maintain and contribute to the development of user and developer documentation.
             \item Solve bug issues, engage in trouble shooting, and interact with the project community.
         \end{itemize}}\\
         & \emph{\textbf{Projects:} Performance API, PAPI}\\
         & \emph{\textbf{Achievements:} designed and developed software components to support performance monitoring on exascale systems, like Aurora and Frontier.}\\

\textsc{09/25/2017 - 05/29/2020} & Postdoctoral Fellow at \textsc{Argonne National Laboratory}, 9700 S Cass Ave, Lemont, IL 60439, USA\\
         & \textbf{\emph{Job Description:}}\\
         & \emph{\begin{itemize}
             \item Work with top researchers on state of the art communication libraries and runtime systems for high-performance computing.
             \item Design and develop advanced software solutions (in C/C++ and Fortran) for communication and runtime libraries.
             \item Test and evaluate software designs, pin point performance bottlenecks and address them.
             \item Find and fix bug issues, write documentation and technical reports.
             \item Publish result and present them to the community at international venues.
             \item Contribute to the research community by serving in Programme Committees of top conferences and reviewing papers for top conferences and journals.
             \item Write training material and present it at Argonne training events (Argonne annual MPI training, Petascale Computing Institute).
         \end{itemize}}\\
         & \emph{\textbf{Projects:} MPICH (high-performance Message Passing Interface implementation)}\\
         & \emph{\textbf{Achievements:} designed and developed new software components to support exascale hardware and memory technologies in systems like Aurora and Frontier.}\\

\textsc{02/18/2011 - 05/15/2017} & Research Engineer at \textsc{Seagate System} Ltd, Springtown Industrial Estate, Londonderry BT48 0LY\\
         & \textbf{\emph{Job Description:}}\\
         & \emph{\begin{itemize}
             \item Design and develop new software libraries and middlewares (in C/C++) for high-performance data storage systems.
             \item Understand state of the art storage and file system technologies, find solutions to alleviate the increasing gap beween storage and memory performance.
             \item Work on European Commission funded exascale research projects.
             \item Collaborate with project partners from industry and academia across Europe.
             \item Work with system architects to explore new designs in storage hardware and software.
         \end{itemize}}\\
         & \emph{\textbf{Projects:} SCALUS (SCALing by mean of Ubiquitous Storage), DEEP-ER (Dynamic Exascale Entry Platform - Extended Reach), SAGE (Percipient StorAGE for exascale data centric computing) and ESiWACE (centre of Excellence in Simulation of Weather And Climate in Europe).}\\
         & \emph{\textbf{Achievements:} Ph.D. in computer science from University of Mainz while conducting research for Seagate.}\\

\textsc{01/20/2009 - 09/01/2010} & Software Developer at \textsc{Sardegna Ricerche, Localita' Piscina Manna, Edificio 2, 09050}, Pula, Italy\\
         & \textbf{\emph{Job Description:}}\\
         & \emph{\begin{itemize}
             \item Design and develop parallel image processing code for the IBM Cell Broadband Engine architecture (CellBE).
             \item Work with other engineers to profile software performance and find bottlenecks through monitoring tools (e.g., oprofile).
             \item Leverage OpenMP parallelism and IBM libspe library to write code for the synergistic processing elements of the Cell processor.
             \item Document code and write technical reports.
         \end{itemize}}\\
         & \emph{\textbf{Projects:} MIACell (Medical Imaging on Cell broadband engine).}\\
         & \emph{\textbf{Achievements:} Developed scale-up prototype of medical imaging software able to help patologists in the identification of carcenogenic cells in paptest slides.}\\
\end{longtable}
%\end{tabular}

%\section{Skills}
%\begin{tabular}{rp{11cm}}
%\textsc{Technical} & Parallel and distributed programming models (MPI, OpenMP, CUDA, HIP, threading) \\
%          & Memory technologies and consistency models \\
%          & C/C++ programming \\
%          & Scripting (Python, Bash, Perl) \\
%          & Software development tools (git, GNU autotools) \\
%          & Network technologies (IB, RoCE) \\
%          & Distributed Parallel I/O and storage technologies \\
%
%\textsc{Soft}      & English language \\
%          & Technical writing \\
%          & Teamwork \\
%\end{tabular}

\section{Scholarships and certificates}
\begin{tabular}{rp{11cm}}
\textsc{2011} & Marie Curie Initial Training Network Certificate \\
          & \footnotesize{European Commission funded Ph.D. programme}
\end{tabular}

%\section{Professional Activities}
%\subsection*{Projects}
%\begin{tabular}{rp{11cm}}
%\textsc{2021 - present} & PAPI: \url{https://www.icl.utk.edu/papi} \\
%\textsc{2017 - 2020}    & MPICH: \url{https://www.mpich.org} \\
%\textsc{2016 - 2017}    & SAGE: \url{http://www.sagestorage.eu} \\
%\textsc{2016 - 2017}    & EsiWace: \url{https://www.esiwace.eu} \\
%\textsc{2013 - 2017}    & DEEP-ER: \url{https://www.deep-projects.eu} \\
%\textsc{2011 - 2013}    & SCALUS (SCALing by mean of Ubiquitous Storage) \\
%\textsc{2009 - 2010}    & MIACell (Medical Image Analysis on Cell broadband engine) \\
%\end{tabular}

%\subsection*{Technical Reviewer for International Journals}
%\begin{tabular}{rp{11cm}}
%    2022 & IEEE Micro \\
%    2021 & Elsevier Journal of Parallel and Distributed Computing (JPDC) \\
%    2019 & Concurrency and Computation: Practices and Experience (CPE) \\
%    2018 & Elsevier Journal of Parallel Computing (PARCO) \\
%    2017 & Elsevier Journal of Parallel and Distributed Computing (JPDC)\\
%    2017 & IEEE Transaction on Parallel and Distributed Systems (TPDS)\\
%\end{tabular}
%
%\subsection*{Technical Reviewer for International Conferences and Workshops}
%\begin{tabular}{rp{11cm}}
%    2023 & IEEE Hot Interconnects Symposium (HOTI) \\
%    2023 & IEEE/ACM International Conference for High Performance Computing,
%    Networking, Storage and Analysis (SC) \\
%    2022 & IEEE Hot Interconnects Symposium (HOTI) \\
%    2020 & IEEE/ACM International Symposium on Cluster, Cloud and Internet Computing (CCGRID)\\
%    2020 & IEEE International Workshop on Accelerators and Hybrid Exascale Systems (AsHES)\\
%    2019 & IEEE International Workshop on Accelerator Programming Using Directives (WACCPD)\\
%    2019 & IEEE Conference on Data Science and Systems (DSS)\\
%    2016 & IEEE/ACM International Symposium on Cluster, Cloud and Grid Computing (CCGRID)\\
%    2015 & IEEE/ACM International Symposium on Cluster, Cloud and Grid Computing (CCGRID)\\
%\end{tabular}
%
%\subsection*{Program Committee Member}
%\begin{tabular}{rp{11cm}}
%    2023 & Program Committee Member of the IEEE Hot Interconnects Symposium (HOTI) \\
%    2023 & Program Committee Member of the IEEE/ACM International Conference for High Performance Computing,
%    Networking, Storage and Analysis (SC) \\
%    2022 & Program Committee Member of the IEEE Hot Interconnects Symposium (HOTI) \\
%    2020 & Program Committee Member of the IEEE/ACM International Symposium on Cluster, Cloud and Internet Computing (CCGRID)\\
%    2020 & Program Committee Member of the IEEE International Workshop on Accelerators and Hybrid Exascale Systems (AsHES)\\
%    2019 & Program Committee Member of the IEEE International Workshop on Accelerator Programming Using Directives (WACCPD)\\
%    2019 & Program Committee Member of the IEEE Conference on Data Science and Systems (DSS)\\
%\end{tabular}
%
%\subsection*{Invited Talks}
%\begin{tabular}{rp{11cm}}
%    2019 & The 9th IEEE International Workshop on Accelerators and Hybrid Exascale Systems: Evaluating the Impact of High Bandwidth Memory on MPI Communciation\\
%\end{tabular}
%
%\subsection*{Tutorials and Talks}
%\begin{tabular}{rp{11cm}}
%    2022 & VI-HPS Tuning Workshop: PAPI (Performance API) Introduction \& Overview \\
%    2019 & Yearly Argonne MPI Tutorial: Hybrid Programming Models, MPI+Accelerators (GPUs) \\
%    2019 & National Center for Supercomputing Applications (NCSA), Petascale Institute: MPI Collectives, MPI Shared Memory and MPI+Accelerators\\
%\end{tabular}
%
%\section{Publications}
%\begin{tabular}{rp{11cm}}
%        2018 & G. Congiu, P. Balaji, "Evaluating the Impact of High-Bandwidth Memory on MPI %
%        Communications" 2018 IEEE International Conference on Computing and Communications (ICCC), %
%        Chengdu China, 2018\\
%        2017 & G. Congiu, M. Grawinkel, F. Padua, J. Morse, T. Süß and A. Brinkmann, "MERCURY: %
%        A Transparent Guided I/O Framework for High Performance I/O Stacks" \textit{2017 IEEE %
%        Euromicro International Conference on Parallel, Distributed and Network-based Processing (PDP)}, %
%        St. Petersburg, 2017. doi: 10.1109/PDP.2017.83\\
%        2016 & G. Congiu, S. Narasimhamurthy, T. Süß and A. Brinkmann, "Improving Collective %
%        I/O Performance Using Non-volatile Memory Devices," \textit{2016 IEEE International Conference %
%        on Cluster Computing (CLUSTER)}, Taipei, 2016, pp. 120-129. %
%        doi: 10.1109/CLUSTER.2016.37\\
%        2014 & G. Congiu, M. Grawinkel, F. Padua, J. Morse, T. Süß and A. Brinkmann, "POSTER: %
%        Optimizing scientific file I/O patterns using advice based knowledge," \textit{2014 IEEE %
%        International Conference on Cluster Computing (CLUSTER)}, Madrid, 2014, pp. 282-283. %
%        doi: 10.1109/CLUSTER.2014.6968763\\
%        2012 & G. Congiu, M. Grawinkel, S. Narasimhamurthy and A. Brinkmann, "One Phase Commit: %
%        A Low Overhead Atomic Commitment Protocol for Scalable Metadata Services," \textit{2012 IEEE %
%        International Conference on Cluster Computing Workshops}, Beijing, 2012, pp. 16-24. %
%        doi: 10.1109/ClusterW.2012.16\\
%\end{tabular}
%
%\bibliographystyle{unsrt}
%\bibliography{bibliography.bib}
\end{document}
